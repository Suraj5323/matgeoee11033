\let\negmpace\undefined
\let\negthickspace\undefined
\documentclass[journal]{IEEEtran}
\usepackage[a5paper, margin=10mm, onecolumn]{geometry}
%\usepackage{lmodern} % Ensure lmodern is loaded for pdflatex
% Include tfrupee package
\setlength{\headheight}{1cm} % Set the height of the header box
\setlength{\headsep}{0mm}     % Set the distance between the header box and the top of the text
\usepackage{xparse}
\usepackage{gvv-book}
\usepackage{gvv}
\usepackage{cite}
\usepackage{amsmath,amssymb,amsfonts,amsthm}
\usepackage{algorithmic}
\usepackage{graphicx}
\usepackage{textcomp}
\usepackage{xcolor}
\usepackage{txfonts}
\usepackage{listings}
\usepackage{enumitem}
\usepackage{mathtools}
\usepackage{gensymb}
\usepackage{comment}
\usepackage[breaklinks=true]{hyperref}
\usepackage{tkz-euclide} 
\usepackage{listings}
% \usepackage{gvv}                                        
\def\inputGnumericTable{}                                 
\usepackage[latin1]{inputenc}                                
\usepackage{color}                                            
\usepackage{array}                                            
\usepackage{longtable}                                       
\usepackage{calc}                                             
\usepackage{multirow}                                         
\usepackage{hhline}                                           
\usepackage{ifthen}                                           
\usepackage{lscape}
\renewcommand{\thefigure}{\theenumi}
\renewcommand{\thetable}{\theenumi}
\setlength{\intextsep}{10pt} % Space between text and floats
\numberwithin{equation}{enumi}
\numberwithin{figure}{enumi}
\renewcommand{\thetable}{\theenumi}
\begin{document}
\bibliographystyle{IEEEtran}
\title{2023-April Session-08-04-2023-shift-1-16-30}
\author{EE24BTECH11033 - KOLLURU SURAJ}
% \maketitle
% \newpage
% \bigskip
{\let\newpage\relax\maketitle}
\begin{enumerate}[start=16]
\item The number of ways, in which 5 girls and 7 boys can be seated at a round table so that no two girls sit together, is
\hfill(April-2023)
\begin{multicols}{4}
    \begin{enumerate}
        \item $7\brak{720}^2$
        \item $720$
        \item $7\brak{360}^2$
        \item $126\brak{5!}^2$
    \end{enumerate}
\end{multicols}
\item Let $f(x) = \frac{\sin x + \cos x - \sqrt{2}}{\sin x - \cos x}$, $x \in \sbrak{0, \pi}-\cbrak{\frac{\pi}{4}}$. Then $f\brak{\frac{7\pi}{12}}f^{\prime\prime}\brak{\frac{7\pi}{12}}$  is equal to
\hfill(April-2023)
\begin{multicols}{4}
    \begin{enumerate}
    \item $\frac{-2}{3}$
    \item $\frac{2}{9}$
    \item $\frac{-1}{3\sqrt3}$
    \item $\frac{2}{3\sqrt{3}}$
    \end{enumerate}
\end{multicols}
\item If the equation of the plane containing the line $x$ + $2y$ + $3z$ - 4 =0, $2x$ + $y$ - $z$ + 5 =0 and perpendicular to the plane $\vec{r}$= $\brak{\hat{i}-\hat{j}}$ + $\lambda\brak{\hat{i}+\vec{j}+\vec{k}}$ +$\mu\brak{\hat{i}-2\hat{j}+3\hat{k}}$ is $ax + by+cz=4$, then $(a-b+c)$ is equal to
\hfill(April-2023)
\begin{multicols}{4}
    \begin{enumerate}
    \item 22
    \item 24
    \item 20
    \item 18
    \end{enumerate}
\end{multicols}
\item Let A =  \myvec{2 & 1 & 0 \\ 1 & 2 & -1 \\ 0 & -1 & 2 } . If  $\abs{\operatorname{adj}\abs{\operatorname{adj}\abs{\operatorname{adj}2A}}}$= $\brak{16}^n$, then $n$ is equal to
\hfill(April-2023)
\begin{multicols}{4}
    \begin{enumerate}
    \item 8
    \item 9
    \item 12
    \item 10
    \end{enumerate}
\end{multicols}
\item Let \( I(x) = \int \frac{\brak{x+1}}{x\brak{1 + x e^x}^2} \, dx, \, x > 0. \) If \( \lim_{x \to \infty} I\brak{x} = 0, \) then \( I\brak{1} \) is equal to
\hfill(April-2023)
\begin{multicols}{2}
    \begin{enumerate}
    \item  $\frac{e+1}{e+2} - \log_e\brak{e+1} $
    \item  $\frac{e+2}{e+1} + \log_e\brak{e+1}$ 
    \item  $\frac{e+2}{e+1} - \log_e\brak{e+1} $
    \item  $\frac{e+1}{e+2} + \log_e\brak{e+1} $
\end{enumerate}
\end{multicols}
\item   Let $A = \sbrak{0, 3, 4, 6, 7, 8, 9, 10}$ and $R$ be the relation defined on $A$ such that $R = \cbrak{(x, y) \in A \times A : x - y \text{ is an odd positive integer or } x - y = 2 }$. The minimum number of elements that must be added to the relation $R$ so that it is a symmetric relation is equal to 
\hfill(April-2023)
\item Let $\sbrak{t}$ denote the greatest integer $\leq t$. If the constant term in the expansion of $\brak{3x^2 - \frac{1}{2x^5}}^7$ is $\alpha$, then $\sbrak{\alpha}$  is equal to 
\hfill(April-2023)
\item Let $\lambda_1, \lambda_2$ be the values of $\lambda$ for which the points 
$\brak{\frac{5}{2}, -1, \lambda}$ and $\brak{-2, 0, 1}$ 
are at equal distance from the plane $2x + 3y - 6z + 7 = 0$. 
If $\lambda_1 > \lambda_2$, then the distance of the point $(\lambda_1 - \lambda_2, \lambda_2, \lambda_1)$ 
from the line $\frac{x - 5}{1} = \frac{y - 1}{2} = \frac{z + 7}{2}$
\hfill(April-2023)
\item If the solution curve of the differential equation $\brak{y - 2 \log_e x}dx + \brak{x \log_e x^2}dy = 0$, $x > 1$ passes through the points $\brak{e, \frac{4}{3}}$ and  $\brak{e^4, \alpha}$, then $\alpha$ is equal to 
\hfill(April-2023)
\item Let \(\hat{a} = 6\hat{i} + 9\hat{j} + 12\hat{k}, \ \hat{b} = \alpha\hat{i} + 11\hat{j} - 2\hat{k}\) and \(\hat{c}\) be vectors such that \(\hat{a} \times \hat{c} = -\hat{a} \times \hat{b}\). If \(\hat{a} \cdot \hat{c} = -12, \ \hat{c} \cdot (\hat{i} - 2\hat{j} + \hat{k}) = 5\), then \(\hat{c} \cdot (\hat{i} + \hat{j} + \hat{k})\) is equal to 
\hfill(April-2023)
\item The largest natural number $n$ such that $3^n$ divides 66! is
\hfill(April-2023)
\item If $a_a$ is the greatest term in the sequence $a_n=\frac{n^3}{n^4 + 147}$, $n$=1,2,3,...., then $a$ is equal to 
\hfill(April-2023)
\item Let the mean and variance of 8 numbers $x$, $y$, 10, 12, 6, 12, 4, 8 be 9 and 9.25 respectively. If $x>y$, then $3x-2y$ is equal to 
\hfill(April-2023)
\item Consider a circle $C_1 : x^2+y^2-4x-2y=\alpha-5$. Let its mirror  image in the line $y=2x+1$ be another circle $C_2 : 5x^2+5y^2 -10fx-10gy+36=0 $.Let $r$ be the radius of $C_2$. Then $\alpha$ + $r$ is equal to
\hfill(April-2023)
\item Let $\sbrak{t}$ denote the greatest integer $\leq t$. Then $\frac{2}{\pi} \int_{\pi/6}^{5\pi/6} \brak{8\sbrak{\cosec x} - 5\sbrak{\cot x} } dx$ is equal to 
\hfill(April-2023)






\end{enumerate}
\end{document}

